% ======================================================================
% TRABALHO: Pontos Turísticos da Baixada Santista - Análise de Similaridade do Cosseno
% AUTOR: Lucas Conceição Marques
% INSTITUIÇÃO: FATEC Rubens Lara
% FORMATAÇÃO: Estilo ABNT 
% ======================================================================

\documentclass[12pt,oneside]{article} % Classe de documento padrão (artigo) -> relatório técnico
% ----------------------------------------------------------------------
% PACOTES DE CONFIGURAÇÃO
% ----------------------------------------------------------------------
\usepackage[brazil]{babel}            % Define o idioma como português 
\usepackage[utf8]{inputenc}           % Codificação de caracteres UTF-8
\usepackage[T1]{fontenc}              % Saída correta de acentuação
\usepackage[a4paper,top=3cm,bottom=2cm,left=3cm,right=2cm]{geometry} % Margens padrão ABNT e tamanho do papel
\usepackage{setspace}                 % Controla espaçamento entre linhas
\usepackage{indentfirst}              % Faz o primeiro parágrafo de cada seção ser indentado
\usepackage{titlesec}                 % Permite personalizar títulos e seções
\usepackage{graphicx}                 % Inserção de figuras
\usepackage{url}                      % Permite incluir URLs formatadas corretamente
\usepackage[hidelinks]{hyperref}      % Cria links clicáveis no PDF sem as bordas coloridas
\usepackage{titling}                  % Personalização do título e capa

% ----------------------------------------------------------------------
% CONFIGURAÇÕES DE FORMATAÇÃO
% ----------------------------------------------------------------------
\onehalfspacing                      % Espaçamento 1,5 entre linhas
\setlength{\parindent}{1.25cm}       % Recuo de parágrafo padrão ABNT
\setlength{\parskip}{0.2cm}          % Espaço entre parágrafos

% ----------------------------------------------------------------------
% INÍCIO DO DOCUMENTO
% ----------------------------------------------------------------------
\begin{document}

% ----------------------------------------------------------------------
% CAPA (formato simples ABNT)
% ----------------------------------------------------------------------

%textbf -> negrito
%textit -> itálico

\begin{center}
    \textbf{FATEC RUBENS LARA} \\[4cm]

    \textbf{LUCAS CONCEIÇÃO MARQUES} \\[6cm]

    \textbf{PONTOS TURÍSTICOS DA BAIXADA SANTISTA:} \\[0.3cm]
    \textbf{Análise de Similaridade do Cosseno} \\[8cm]

    \textbf{Santos} \\
    \textbf{2025}
\end{center}

\newpage

%================================
%            sumário
%=================================

%Gera automaticamente o Sumário com base nas seções
\newpage
\renewcommand{\contentsname}{\centering Sumário} %centraliza o sumário
\tableofcontents
\newpage


% ----------------------------------------------------------------------
% INTRODUÇÃO
% ----------------------------------------------------------------------

\section{Introdução ao Tema}
O tema consiste na análise de similaridade do cosseno aplicada aos pontos turísticos da Baixada Santista. 
Nesse contexto, o usuário terá descrito o tipo de local que deseja visitar. 
Com base em arquivos \texttt{.txt} armazenados em um arquivo ZIP, contendo informações sobre diferentes pontos turísticos, 
o sistema realizará a análise para indicar qual local apresenta maior similaridade com a preferência do usuário.

%\texttt -> destacar arquivos, deixar monoespaçado


% ----------------------------------------------------------------------
% DESENVOLVIMENTO
% ----------------------------------------------------------------------
\section{Desenvolvimento do Trabalho}

\subsection{Importação das Bibliotecas e Ferramentas}
O programa se inicia com a importação das seguintes bibliotecas: 
\textbf{Numpy} (para manipulação de vetores e cálculos numéricos), 
\textbf{NLTK} (essencial para o Processamento de Linguagem Natural e utilizada para obter as \textit{stopwords}). 
O comando \texttt{nltk.download('stopwords')} baixa o conjunto de dados necessário.

Em seguida, importam-se: \textbf{OS} (para interagir com o sistema operacional, manipulando caminhos de arquivo), 
\textbf{Zipfile} (para manipular arquivos ZIP), 
\textbf{TfidfVectorizer} (do \texttt{Sklearn}, para conversão de textos em vetores TF-IDF) 
e \textbf{cosine\_similarity} (para o cálculo da similaridade do cosseno).

\begin{figure}[h!] %h -> here(aqui)
    \centering
    \includegraphics[width=0.9\textwidth]{figura1.png} % ajusta o tamanho
    \caption{Trecho inicial do código em Python com importações e setup.} %legenda numerada
\end{figure}



% ----------------------------------------------------------------------
% SEÇÃO 1
% ----------------------------------------------------------------------
\section{1. Pontos Turísticos}

%h -> here(aqui)
%  width  -> ajusta o tamanho
%\caption -> legenda numerada

\begin{figure}[h!] 
    \centering
    \includegraphics[width=0.9\textwidth]{figura2.png} 
    \caption{Trecho de código referente ao upload do arquivo ZIP.} 
\end{figure}

\subsection{1.1 Upload}
\textbf{Descrição:} Nesse ponto, o módulo \texttt{files} (do Google Colab) permite interações com o navegador. 
O comando \texttt{files.upload()} executa um pop-up solicitando o upload do arquivo \texttt{pontos\_turisticos.zip} 
e armazena seu conteúdo na variável \texttt{uploaded}.



\subsection{1.2 Extrair Conteúdo do ZIP}
\textbf{Descrição:} A variável \texttt{zip\_filename} define o nome do arquivo ZIP esperado. 
A estrutura condicional verifica se o arquivo foi carregado com êxito. 
Se sim, \texttt{with zipfile.ZipFile(zip\_filename, 'r')} abre o arquivo em modo de leitura e 
\texttt{zip\_ref.extractall('.')} extrai todo o conteúdo do ZIP para o diretório atual. 
Caso ocorra erro, uma mensagem de aviso será exibida.

\begin{figure}[h!]
    \centering
    \includegraphics[width=0.9\textwidth]{figura3.png} 
    \caption{Código para extração do conteúdo do arquivo ZIP.} 
\end{figure}



\subsection{1.3 Carregar os Arquivos}
\textbf{Descrição:} Após a extração, define-se a variável \texttt{descrição}, que será a consulta (\textit{query}) de busca. 
Cria-se um dicionário \texttt{pontos\_dados} para armazenar o nome e o texto de cada ponto turístico. 
Um laço percorre os arquivos do diretório, filtrando apenas os com extensão \texttt{.txt}. 
A função \texttt{os.path.join()} constrói o caminho completo e o conteúdo é lido e armazenado. 
No dicionário, o nome do ponto é a chave e o texto é o valor. Listas são criadas com os nomes e textos, formando \texttt{compare\_list}, que contém a descrição do usuário e os textos dos pontos turísticos.

\begin{figure}[h!]
    \centering
    \includegraphics[width=0.9\textwidth]{figura4.png}
    \caption{Trecho do código responsável por carregar e ler os arquivos de texto.}
\end{figure}

\vspace{2cm}
%cria espaço vertical de 2cm

% ----------------------------------------------------------------------
% STOPWORDS
% ----------------------------------------------------------------------

\section{Stopwords}
As \textit{stopwords} são importadas de \texttt{nltk.corpus}. 
Com \texttt{set()}, obtém-se a lista padrão em português. 
Em seguida, cria-se uma lista personalizada (\texttt{custom\_stop}) com palavras genéricas (\textit{local}, \textit{visita}, etc.) e termos adicionais. 
A lista final \texttt{br\_stop} é o resultado da união entre a lista padrão e a personalizada.

\begin{figure}[h!]
    \centering
    \includegraphics[width=0.95\textwidth]{figura5.png}
    \caption{Código de criação da lista de \textit{stopwords} padrão e personalizada.}
\end{figure}
% ----------------------------------------------------------------------
% VETORIZAÇÃO
% ----------------------------------------------------------------------

\section{Vetorização do Texto}
Com as stopwords definidas, \texttt{TfidfVectorizer()} é usado para removê-las durante a vetorização. 
O método \texttt{fit\_transform()} analisa e transforma os textos em vetores.

\begin{figure}[h!]
    \centering
    \includegraphics[width=0.95\textwidth]{figura6.png}
    \caption{Código responsável pela vetorização dos textos usando TF-IDF.}
\end{figure}
% ----------------------------------------------------------------------
% SIMILARIDADE
% ----------------------------------------------------------------------


\section{Impressão das Similaridades}
A biblioteca \texttt{cosine\_similarity} calcula a similaridade de cosseno entre os vetores da descrição do usuário e os dos pontos turísticos. 
O resultado é transformado em uma lista simples com \texttt{flatten()}. 
Com \texttt{numpy.arccos()} e \texttt{numpy.degrees()}, obtém-se o ângulo em graus. 
O programa imprime os nomes, similaridades e ângulos, indicando o ponto mais próximo do desejado.

\begin{figure}[h!]
    \centering
    \includegraphics[width=0.95\textwidth]{figura7.png}
    \caption{Código para o cálculo da similaridade de cosseno entre o texto do usuário e os pontos turísticos.}
\end{figure}

% ----------------------------------------------------------------------
% RESULTADOS
% ----------------------------------------------------------------------



\section{Output}
Analisando os dados obtidos, observa-se que o ponto turístico com maior taxa de similaridade e menor ângulo corresponde às praias, 
enquanto os mais distantes são museus e centros históricos.

\begin{figure}[h!]
    \centering
    \includegraphics[width=0.95\textwidth]{figura8.png}
    \caption{imprime o resultado.}
\end{figure}



\section{Ponto Turístico Recomendado}

\begin{figure}[h!]
    \centering
    \includegraphics[width=0.95\textwidth]{figura9.png}
    \caption{Trecho final do código que identifica o ponto turístico mais similar.}
\end{figure}


\begin{center}
\begin{tabular}{|p{14cm}|} %tabular -> ambiente de tabela, p{14} -> cria uma única coluna de largura fixa de 14cm
\hline 
\textbf{✅ Praia da Enseada é o ponto turístico mais similar à sua descrição com 76.74 graus.} \\[0.3cm]
A praia mais extensa do Guarujá, com cerca de 6 km de comprimento. É famosa por suas águas calmas e rasas, o que a torna perfeita para famílias com crianças e para a prática de esportes náuticos como stand up paddle e caiaque. 
A orla é repleta de hotéis, restaurantes e quiosques com boa infraestrutura, sendo um dos pontos mais movimentados da cidade. \\ 
\hline %cria uma linha horizontal
\end{tabular}
\end{center}

Para finalizar, a função \texttt{argmax()} localiza o ID com maior similaridade e obtém o nome e texto do ponto turístico mais semelhante — neste caso, a \textbf{Praia da Enseada}, com taxa de similaridade de 0.229 e ângulo de 76.74 graus.


\vspace{2cm}

% ----------------------------------------------------------------------
% REFERÊNCIAS
% ----------------------------------------------------------------------
\section{Referências}

MATEMÁTICA COM PROFA JAQUELINE SILVA. \textit{Introdução ao LaTeX e ao Overleaf | Aula 01}. 
YouTube, 1 dez. 2019. Disponível em: \url{https://youtu.be/Y1vdXYttLSA?si=XPlbguzwkShXvtXF}. Acesso em: 25 out. 2025.

SIMPLE TECH. \textit{Python Cosine Similarity - sklearn TfidfVectorizer and Spacy en\_core\_web\_lg}. 
YouTube, 9 ago. 2022. Disponível em: \url{https://youtu.be/DIxxz_DvqLA?si=syqwoDYpep8Zyd-0}. Acesso em: 25 out. 2025.

\end{document}
